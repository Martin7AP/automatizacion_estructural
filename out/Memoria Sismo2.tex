\documentclass{article}%
\usepackage[T1]{fontenc}%
\usepackage[utf8]{inputenc}%
\usepackage{lmodern}%
\usepackage{textcomp}%
\usepackage{lastpage}%
\usepackage{geometry}%
\geometry{left=2.5cm,top=1.5cm}%
\usepackage[dvipsnames]{xcolor}%
\usepackage{array}%
\usepackage{colortbl}%
\usepackage{graphicx}%
\usepackage{caption}%
\usepackage{slashbox}%
\usepackage{amsmath}%
\usepackage{xcolor}%
\usepackage{multirow}%
\usepackage{tcolorbox}%
\usepackage{booktabs}%
\usepackage{ragged2e}%
%
\graphicspath{ {D:/CARPETAS W10/Escritorio/GITHUB/automatizacion_estructural/} }%
%
\begin{document}%
\normalsize%
\subsection{Factor de Zona}%
\label{subsec:FactordeZona}%
%


\begin{table}[ht!]%
\begin{minipage}{0.55\textwidth}%
\caption{Factor de zona}%
\begin{tabular}{|>{\centering\arraybackslash}m{3.75cm}|>{\centering\arraybackslash}m{3.75cm}|}%
\hline%
\multicolumn{2}{|c|}{\textbf{FACTOR DE ZONA SEGÚN E{-}030}}\\%
\hline%
\textbf{ZONA}&\textbf{Z}\\%
\hline%
Zona&Z\\%
\hline%
\end{tabular}%
\end{minipage}%
\begin{minipage}{0.35\textwidth}%
\begin{center}%
\includegraphics[width=4cm]{mapa_zona}%
\end{center}%
\end{minipage}%
\caption*{Fuente: E-30 (2018)}%
\end{table}

%
\subsubsection{Factor de suelo}%
\label{ssubsec:Factordesuelo}%
%


\begin{table}[ht!]%
\centering%
\caption{Factor de zona}%
\begin{tabular}{|>{\centering\arraybackslash}m{3.75cm}|>{\centering\arraybackslash}m{2cm}|>{\centering\arraybackslash}m{2cm}|>{\centering\arraybackslash}m{2cm}|>{\centering\arraybackslash}m{2cm}|}%
\hline%
\multicolumn{5}{|c|}{\textbf{FACTOR DE SUELO SEGÚN E{-}030}}\\%
\hline%
\backslashbox{\textit{\textbf{ZONA}}}{\textit{\textbf{SUELO}}}&\textbf{S0}&\textbf{S1}&\textbf{S2}&\textbf{S3}\\%
\hline%
4\cellcolor[rgb]{ .949,  .949,  .949} &0.80\cellcolor[rgb]{ .949,  .949,  .949} &1.00\cellcolor[rgb]{ .949,  .949,  .949} &\textcolor[rgb]{ 1,  0,  0}{\textbf{1.05}}\cellcolor[rgb]{ .949,  .949,  .949} \cellcolor[rgb]{ .949,  .949,  .949} &1.10\cellcolor[rgb]{ .949,  .949,  .949} \\%
\hline%
3&0.80&1.00&1.15\cellcolor[rgb]{ .949,  .949,  .949} &1.20\\%
\hline%
2&0.80&1.00&1.20\cellcolor[rgb]{ .949,  .949,  .949} &1.40\\%
\hline%
1&0.80&1.00&1.60\cellcolor[rgb]{ .949,  .949,  .949} &2.00\\%
\hline%
\end{tabular}%
\caption*{Fuente: E-30 (2018)}%
\end{table}

%
\subsubsection{Periodos de suelo}%
\label{ssubsec:Periodosdesuelo}%
%


\begin{table}[ht!]%
\centering%
\caption{Periodos de suelo}%
\begin{tabular}{|>{\centering\arraybackslash} m{2cm}|>{\centering\arraybackslash}m{2cm}|>{\centering\arraybackslash}m{2cm}|>{\centering\arraybackslash}m{2cm}|>{\centering\arraybackslash}m{2cm}|}%
\cline{2-5}%
\multicolumn{1}{r|}{}&\multicolumn{4}{c|}{\textbf{PERIODO "Tp" y "Tl" SEGÚN E-030}}\\%
\cline{2-5}%
\multicolumn{1}{r|}{}&\multicolumn{4}{c|}{\textit{\textbf{Perfil de suelo}}}\\%
\cline{2-5}%
\multicolumn{1}{r|}{}&\textbf{S0}&\textbf{S1}&\textbf{S2}&\textbf{S3}\\%
\hline%
Tp&0.30&0.40&\textcolor[rgb]{ 1,  0,  0}{\textbf{0.60}}\cellcolor[rgb]{ .949,  .949,  .949} &1.00\\%
\hline%
Tl&3.00&2.50&\textcolor[rgb]{ 1,  0,  0}{\textbf{2.00}}\cellcolor[rgb]{ .949,  .949,  .949} &1.60\\%
\hline%
\end{tabular}%
\caption*{Fuente: E-30 (2018)}%
\end{table}

%
\subsubsection{Sistema Estructural}%
\label{ssubsec:SistemaEstructural}%
Después de realizar el análisis sísmico se determino que los sistemas estructurales en X, Y son:%
%


\begin{table}[ht!]%
\caption{coeficiente básico de reducción}%
\begin{tabular}{|>{\arraybackslash}m{10cm}| >{\centering\arraybackslash}m{4cm}|}%
\hline%
\multicolumn{2}{|c|}{\textbf{SISTEMAS ESTRUCTURALES}}\\%
\hline%
\textbf{Sistema Estructural}&\multicolumn{1}{m{4cm}|}{\textbf{Coeficiente Básico de Reducción Ro}}\\%
\hline%
\multicolumn{2}{|l|}{\textbf{Acero:}}\\%
\hline%
Porticos Especiales Resistentes a Momento (SMF)&8\\%
\hline%
Porticos Intermedios Resistentes a Momento (IMF)&5\\%
\hline%
Porticos Ordinarios Resistentes a Momento (OMF)&4\\%
\hline%
Porticos Ordinarios Resistentes a Momento (OMF)&7\\%
\hline%
Porticos Ordinarios Concentricamente Arrriostrados (OCBF)&4\\%
\hline%
Porticos Excentricamente Arriostrados (EBF)&8\\%
\hline%
\multicolumn{2}{|l|}{\textbf{Concreto Armado:}}\\%
\hline%
Porticos&8\\%
\hline%
Dual&7\\%
\hline%
De muros estructurales&6\\%
\hline%
Muros de ductilidad limitada&4\\%
\hline%
\textbf{Albañilería Armada o Confinada}&3\\%
\hline%
\textbf{Madera}&7\\%
\hline%
\end{tabular}%
\caption*{Fuente: E-30 (2018)}%
\end{table}

%
\subsubsection{Factor de Amplificación sísmica}%
\label{ssubsec:FactordeAmplificacinssmica}%
%
Se determina según el artículo 11 de la E{-}30%
\setlength{\jot}{0.5cm}%


\begin{figure}[h!]%
\caption{Factor de amplificación}%
\begin{minipage}{0.5\textwidth}%

    \begin{align*}
        &T< T_{P}         &   C&=2,5\cdot\left ( \frac{T_{P}}{T} \right )\\
        &T_{P}< T< T_{L}  &   C&=2,5\cdot\left ( \frac{T_{P}}{T} \right )\\
        &T> T_{L}         &   C&=2,5\cdot\left ( \frac{T_{P}\;T_{L}}{T^{2}} \right )
    \end{align*}%
\end{minipage}%
\begin{minipage}{0.4\textwidth}%
\centering%
\includegraphics[width=6.5cm]{Amplificacion}%
\end{minipage}%
\caption*{Fuente: Muñoz (2020)}%
\end{figure}

%
\subsubsection{Factor de Importancia}%
\label{ssubsec:FactordeImportancia}%
%


\begin{table}[h!]%
\centering%
\caption{Factor de Uso o Importancia}%
\begin{tabular}{|>{\arraybackslash}m{3cm}|m{8cm}|>{\arraybackslash}m{2.8cm}|}%
\hline%
\multicolumn{3}{|c|}{\textbf{CATEGORIA DE LA EDIFICACION}}\\%
\hline%
\multicolumn{1}{|c|}{\textbf{CATEGORIA}}&\multicolumn{1}{|c|}{\textbf{DESCRIPCION}}&\multicolumn{1}{|c|}{\textbf{FACTOR U}}\\%
\hline%
\multirow{2}[4]{3cm}{A Edificaciones Escenciales}&A1: Establecimiento del sector salud (públicos y privados) del segundo y tercer nivel, según lo normado por el ministerio de salud.&\multicolumn{1}{>{\centering\arraybackslash}m{2.8cm}|}{Con aislamiento 1.0 y sin aislamiento 1.5.}\\%
\cline{2-3}%
&A2: Edificaciones escenciales para el manejo de las emergencias, el funcionamiento del gobierno y en general aquellas que puedan servir de refugio después de un desastre.&\multicolumn{1}{>{\centering\arraybackslash}m{2.8cm}|}{1.50}\\%
\hline%
B Edificaciones Importantes &Edificaciones donde se reúnen gran cantidad de personas tales como cines, teatros, estadios, coliseos, centros comerciales, terminales de buses de pasajeros, establecimientos penitenciarios, o que guardan patrimonios valiosos como museos y bibliotecas.&\multicolumn{1}{>{\centering\arraybackslash}m{2.8cm}|}{1.30}\\%
\hline%
C Edificaciones Comunes\cellcolor[rgb]{ .949,  .949,  .949}&Edificaciones comunes tales como: viviendas, oficinas, hoteles, restaurantes, depósitos e instalaciones industriales cuya falla no acarree peligros adicionales de incendios o fugas de contaminantes.\cellcolor[rgb]{ .949,  .949,  .949}&\multicolumn{1}{>{\centering\arraybackslash}m{2.8cm}|}{\textcolor[rgb]{ 1,  0,  0}{\textbf{1.00}}\cellcolor[rgb]{ .949,  .949,  .949}}\\%
\hline%
D Edificaciones temporales&Construcciones provisionales para depósitos, casetas y otras similares.&\multicolumn{1}{>{\centering\arraybackslash}m{2.8cm}|}{A criterio del proyectista}\\%
\hline%
\end{tabular}%
\caption*{Fuente: E-30 (2018)}%
\end{table}

%
\subsection{Análisis modal Art. 26.1 E{-}030}%
\label{subsec:AnlisismodalArt.26.1E{-}030}%
\begin{tcolorbox}[colback=gray!5!white,colframe=Maroon!75!black,fonttitle=\bfseries,title=Art. 26.1.1]%
\textit{En cada dirección se consideran aquellos modos de vibración cuya suma de masas efectivas sea por lo menos el 90\% de la masa total, pero se toma en cuenta por lo menos los tres primeros modos predominantes en la dirección de análisis.}%
\end{tcolorbox}%
\begin{tcolorbox}[colback=gray!5!white,colframe=Maroon!75!black,fonttitle=\bfseries,title=Art. 26.1.2]%
\textit{En cada dirección se consideran aquellos modos de vibración cuya suma de masas efectivas sea por lo menos el 90\% de la masa total, pero se toma en cuenta por lo menos los tres primeros modos predominantes en la dirección de análisis.}%
\end{tcolorbox}%
%


\begin{table}[h!]%
\extrarowheight = -0.3ex%
\renewcommand{\arraystretch}{1.3}%
\centering%
\caption{Periodos y porcentajes de masa participativa}%
\begin{tabular}{cccccccc}
\toprule
Mode & Period & UX & UY & RZ & SumUX & SumUY & SumRZ \\
\midrule
1 & 0.272 & 0.800 & 0.000 & 0.014 & 0.800 & 0.000 & 0.014 \\
2 & 0.153 & 0.012 & 0.136 & 0.687 & 0.812 & 0.136 & 0.701 \\
3 & 0.135 & 0.003 & 0.737 & 0.136 & 0.815 & 0.874 & 0.837 \\
4 & 0.094 & 0.057 & 0.000 & 0.035 & 0.872 & 0.874 & 0.872 \\
5 & 0.072 & 0.108 & 0.000 & 0.007 & 0.980 & 0.874 & 0.879 \\
6 & 0.052 & 0.000 & 0.076 & 0.025 & 0.981 & 0.950 & 0.904 \\
7 & 0.047 & 0.001 & 0.019 & 0.079 & 0.982 & 0.969 & 0.983 \\
8 & 0.043 & 0.018 & 0.000 & 0.000 & 1.000 & 0.969 & 0.983 \\
9 & 0.039 & 0.000 & 0.021 & 0.004 & 1.000 & 0.989 & 0.987 \\
10 & 0.033 & 0.000 & 0.004 & 0.001 & 1.000 & 0.994 & 0.988 \\
11 & 0.029 & 0.000 & 0.006 & 0.000 & 1.000 & 1.000 & 0.988 \\
12 & 0.029 & 0.000 & 0.000 & 0.012 & 1.000 & 1.000 & 1.000 \\
\bottomrule
\end{tabular}
%
\end{table}

%
\subsubsection{Irregularidad de Rigidez{-}Piso Blando}%
\label{ssubsec:IrregularidaddeRigidez{-}PisoBlando}%
\begin{tcolorbox}[colback=gray!5!white,colframe=cyan!75!black,fonttitle=\bfseries,title=Tabla N°9 E-030]%
Existe irregularidad de rigidez cuando, en cualquiera de las direcciondes de análisis, en un entrepiso la rigidez lateral es menor que 70\% de la rigidez lateral del entrepiso inmediato superior, o es menor que 80\% de la rigidez lateral promedio de los tres niveles superiores adyacentes. 
 Las rigideces laterales pueden calcularse como la razón entre la fuerza cortante del entrepiso y el correspondiente desplazamiento relatibo en el centro de masas, ambos evaluados para la misma condición de carga %
\end{tcolorbox}%
\begin{tcolorbox}[colback=gray!5!white,colframe=cyan!75!black,fonttitle=\bfseries,title=Tabla N°9 E-030]%

Existe irregularidad extrema de rigidez cuando, en cualquiera de las direcciones de análisis, en un entrepiso la rigidez lateral es menor que 60\% de la rigidez lateral del entrepiso inmediato superior, o es menor que 70\% de la rigidez lateral promedio de los tres niveles superiores adyacentes.
Las rigideces laterales pueden calcularse como la razon entre la fuerza cortante del entrepiso y el correspondiente desplazamiento relativo en el centro de masas, ambos evaluados para la misma condición de carga.%
\end{tcolorbox}%
%


\begin{table}[h!]%
\centering%
\caption{Irregularidad de rigidez}%
\begin{tabular}{cccccccc}
\toprule
Story & OutputCase & VX & VY & Rigidez Lateral(k) & 70\%k previo & 80\%Prom(k) & is\_reg \\
\midrule
Story4 & SDx Max & 6.946 & 0.443 & 1844.167 &  &  & Regular \\
Story3 & SDx Max & 54.365 & 3.220 & 58547.273 & 1290.917 &  & Regular \\
Story2 & SDx Max & 94.703 & 5.744 & 78678.082 & 40983.091 &  & Regular \\
Story1 & SDx Max & 114.666 & 7.165 & 110232.308 & 55074.658 & 37085.206 & Regular \\
\bottomrule
\end{tabular}
%
\end{table}

%


\begin{table}[h!]%
\centering%
\caption{Irregularidad de rigidez}%
\begin{tabular}{cccccccc}
\toprule
Story & OutputCase & VX & VY & Rigidez Lateral(k) & 70\%k previo & 80\%Prom(k) & is\_reg \\
\midrule
Story4 & SDy Max & 2.600 & 5.576 & 9834.568 &  &  & Regular \\
Story3 & SDy Max & 1.366 & 48.373 & 87315.523 & 6884.198 &  & Regular \\
Story2 & SDy Max & 5.470 & 90.187 & 118823.979 & 61120.866 &  & Regular \\
Story1 & SDy Max & 7.165 & 113.995 & 139188.278 & 83176.785 & 57593.085 & Regular \\
\bottomrule
\end{tabular}
%
\end{table}

%
\subsubsection{Irregularidad Torsional}%
\label{ssubsec:IrregularidadTorsional}%
\begin{tcolorbox}[colback=gray!5!white,colframe=cyan!75!black,fonttitle=\bfseries,title=Tabla N°9 E-030]%
Existe irregularidad torsional cuando, en cualquiera de las direcciones de análisis el desplazamiento relativo de entrepiso en un edificion ($\Delta_{max}$) en esa dirección, calculado incluyendo excentricidad accidental, es mayor que 1,3 veces el desplazamineto relativo promedio de los extremos del mismo entrepiso para la condicion de carga ($\Delta_{prom}$). 
 Este crriterio sólo se aplica en edificios con diafragmas rígidos y sólo si el máximo desplazamiento relativo de entrepiso es mayor que 50\% del desplazamiento permisible indicado en la Tabla N° 11%
\end{tcolorbox}%
\begin{tcolorbox}[colback=gray!5!white,colframe=cyan!75!black,fonttitle=\bfseries,title=Tabla N°9 E-030]%
Existe irregularidad torsional cuando, en cualquiera de las direcciones de análisis el desplazamiento relativo de entrepiso en un edificion ($\Delta_{max}$) en esa dirección, calculado incluyendo excentricidad accidental, es mayor que 1,3 veces el desplazamineto relativo promedio de los extremos del mismo entrepiso para la condicion de carga ($\Delta_{prom}$). 
 Este crriterio sólo se aplica en edificios con diafragmas rígidos y sólo si el máximo desplazamiento relativo de entrepiso es mayor que 50\% del desplazamiento permisible indicado en la Tabla N° 11%
\end{tcolorbox}%
%


\begin{table}[ht!]%
\centering%
\caption{Irregularidad Torsional}%
\resizebox{\textwidth}{!}{%
\begin{tabular}{cccccccccc}
\toprule
Story & OutputCase & Direction & Max Drift & Avg Drift & Ratio & Height & Drifts & < Driftmax/2 & Es Regular \\
\midrule
Story4 & SDx Max & X & 0.003045 & 0.003029 & 1.005 & 2.3 & 0.009135 & False & Regular \\
Story3 & SDx Max & X & 0.003497 & 0.003316 & 1.055 & 3 & 0.010491 & False & Regular \\
Story2 & SDx Max & X & 0.003958 & 0.003565 & 1.11 & 3 & 0.011874 & False & Regular \\
Story1 & SDx Max & X & 0.003226 & 0.002729 & 1.182 & 3.3 & 0.009678 & False & Regular \\
\bottomrule
\end{tabular}
}%
\end{table}

%


\begin{table}[ht!]%
\centering%
\caption{Irregularidad Torsional}%
\resizebox{\textwidth}{!}{%
\begin{tabular}{cccccccccc}
\toprule
Story & OutputCase & Direction & Max Drift & Avg Drift & Ratio & Height & Drifts & < Driftmax/2 & Es Regular \\
\midrule
Story4 & SDy Max & X & 0.000885 & 0.000837 & 1.057 & 2.3 & 0.002655 & False & Regular \\
Story4 & SDy Max & Y & 0.000646 & 0.000626 & 1.031 & 2.3 & 0.001938 & True & Regular \\
Story3 & SDy Max & X & 0.000604 & 0.000333 & 1.813 & 3 & 0.001812 & True & Regular \\
Story3 & SDy Max & Y & 0.000631 & 0.000597 & 1.057 & 3 & 0.001893 & True & Regular \\
Story2 & SDy Max & X & 0.00075 & 0.000405 & 1.854 & 3 & 0.002250 & True & Regular \\
Story2 & SDy Max & Y & 0.000866 & 0.000814 & 1.064 & 3 & 0.002598 & False & Regular \\
Story1 & SDy Max & X & 0.000706 & 0.000377 & 1.873 & 3.3 & 0.002118 & True & Regular \\
Story1 & SDy Max & Y & 0.000966 & 0.000883 & 1.093 & 3.3 & 0.002898 & False & Regular \\
\bottomrule
\end{tabular}
}%
\end{table}

%
\section{Análisis Sísmico}%
\label{sec:AnlisisSsmico}%
None%
None%
None%
None%
None%
None%
None%
None%
None

%
\end{document}